% Grundgröße 12pt, zweiseitig
\documentclass[12pt,a4paper,twoside,parskip=half-,headsepline,headinclude]{scrreprt}
% Seitenköpfe automatisch
\usepackage[headsepline,automark]{scrpage2}
% Sprachpaket für Deutsch (Umlaute, Trennung, deutsche Überschriften)
\usepackage[ngerman]{babel}
\usepackage{blindtext}
%Graphikeinbindung, Hyperref (alles klickbar, Bookmarks),
\usepackage{graphicx, hyperref, amssymb}
% Umlaute - nochmal, die oben gehen nicht...
\usepackage[utf8]{inputenc}
% Math (Symbole aus AmsTeX)
\usepackage{amsmath}
% Einbindung von Code
\usepackage{listings}
% Farben
\usepackage{color}
% Fun with neural nets
\usepackage{tikz}

% Farben (für Codelistings)
\definecolor{mygreen}{rgb}{0,0.6,0}
\definecolor{mygray}{rgb}{0.3,0.3,0.3}
\definecolor{mymauve}{rgb}{0.58,0,0.82}

% Genereller Style für Codelistings
\lstset{ %
  backgroundcolor=\color{white},
  breakatwhitespace=false,
  breaklines=true,
  captionpos=b,
  commentstyle=\color{mygreen},
  escapeinside={\%*}{*)},
  extendedchars=true,
  keepspaces=true,
  keywordstyle=\color{blue},
  language=Octave,
  morekeywords={*, \_\_shared\_\_, until, each, synchronize threads},
  numbers=left,
  numbersep=5pt,
  numberstyle=\tiny\color{mygray},
  rulecolor=\color{black},
  showspaces=false,                
  showstringspaces=false,
  showtabs=false,
  stepnumber=1,
  stringstyle=\color{mymauve},
  tabsize=2
}

% Optionen für neurale Netze
\def\layersep{2.5cm}

% Festlegung Kopf- und Fußzeile     
\defpagestyle{meinstil}{%
	{\headmark \hfill}
	{\hfill \headmark}
	{\hfill \headmark\hfill }
	(\textwidth,.4pt)
}{%
	(\textwidth,.4pt)
	{\pagemark\hfill Alexander-Sebastian Clauß}
	{Version 1.0 vom \today \hfill \pagemark}
	{Version 1.0 vom \today\hfill\pagemark} 
}
\pagestyle{meinstil} 

\raggedbottom
\renewcommand{\topfraction}{1}
\renewcommand{\bottomfraction}{1}

 
%%%%%%%%%%%%%%%%%%%%%%%%%%%%%%%%%%%%%%%%%%%%%%%%%%%%%%%%%%%%%%%%%%%%%%%%%
%							hier gehts los								%
%%%%%%%%%%%%%%%%%%%%%%%%%%%%%%%%%%%%%%%%%%%%%%%%%%%%%%%%%%%%%%%%%%%%%%%%%

\begin{document}    
  \thispagestyle{empty} % Titelseite
\includegraphics[width=0.2\textwidth]{images/hsh_wortmarke.pdf}
{ \sffamily
  \vfill
  {
    \Huge\bfseries Unsupervised Image Clustering
  }
  \bigskip

  {
    \Large 
  	Alexander-Sebastian Clauß \\[2ex]
 	Master-Arbeit im Studiengang "`Angewandte Informatik"' 
    \\[5ex]
    \today 
  } 
}
\vfill
\hfill

\includegraphics[height=0.3\paperheight]{images/hsh_logo.pdf} 

\vspace*{-3cm}

\newpage \thispagestyle{empty}
\begin{tabular}{ll}
{\bfseries\sffamily Autor} & Alexander-Sebastian Clauß \\ 
            			   & Matrikelnummer: 1381164 \\
            			   & alexander-sebastian.clauss@hs-hannover.de \\[5ex]
{\bfseries\sffamily Erstprüferin:} & Prof. Dr. Frauke Sprengel \\
          						   & Abteilung Informatik, Fakultät IV \\
         						   & Hochschule Hannover \\
        						   & frauke.sprengel@hs-hannover.de \\[5ex]
{\bfseries\sffamily Zweitprüfer:}  & Maximilian Zubke \\
          						   & Abteilung Information und Kommunikation, Fakultät III \\
         						   & Hochschule Hannover \\
        						   & maximilian.zubke@hs-hannover.de
\end{tabular}

\vfill

% fett und zentriert in der minipage
\begin{center} \sffamily\bfseries Selbständigkeitserklärung \end{center}

Hiermit erkläre ich, dass ich die eingereichte Master-Arbeit
selbständig und ohne fremde Hilfe verfasst, andere als die von mir angegebenen Quellen und Hilfsmittel nicht benutzt und die den benutzten Werken wörtlich oder
inhaltlich entnommenen Stellen als solche kenntlich gemacht habe. 
\vspace*{7ex}

Hannover, den \today \hfill Unterschrift

\pdfbookmark[0]{Inhalt}{contents}

\tableofcontents
\listoffigures
\listoftables

\chapter{Einleitung}

\begin{enumerate}
	\item Beschreibung der Kapitel
\end{enumerate}

\section{Motivation}

\begin{enumerate}
	\item Ermöglichen der  Bildsuche im Bestand
\end{enumerate}

\section{Ziele der Arbeit}

\begin{enumerate}
	\item Schaffung einer Basis
	\item Performancebetrachtung / -gewinn GPU
	\item Erweiterung auf medizinische Anwendung (MHH)
\end{enumerate}
\chapter{Grundlagen}

IMAGES

Im Grundlagenkapitel wird zunächst erläutert an welchen Eigenschaften der Bilder wir interessiert sind, den sogenannten Features. 

DETECTION / EXTRACTION

Im Bereich der Computer Vision gibt es zahlreiche Detektoren, die sich zur Erkennung verschiedener Features eignen. Anschließend werden verschiedene Verfahren der Feature Extraktion vorgestellt, um die Daten für den Lernalgorithmus zu gewinnen. 

REDUCTION

Als Lernalgorithmus wird hier ein Autoencoder verwendet, eine spezielle Art des neuronalen Netzes. Ein Autoencoder reduziert die Dimensionen eines Features, sodass eine kompaktere Darstellung für den Vergleich resultiert.

GPU

Um die Parallelität neuronaler Netze weiter auszunutzen, soll die Ausführung durch eine GPU erfolgen. Hierfür wird aus dem Bereich des GPGPU Programming die Sprache cuda von Nvidia  


\section{Bilder und Features}

In Bild kann auf viele Weisen dargestellt werden. Eine sehr intuitive Darstellung ist das Bild als Matrix. Diese Form eignet sich aber nur sehr eingeschränkt, wenn Bilder anhand von allgemeinen Merkmalen verglichen werden sollen. Ist so zum Beispiel auf zwei Bildern dasselbe Fahrzeug zu erkennen aber aus zwei verschiedenen Perspektiven und in anderen Lichtverhältnissen, so soll dies vom Programm erkannt werden. Zu diesem Zweck werden die Charakteristika, die sogenannten Features, in dem Bild gesucht. Bei einem Feature handelt es sich dabei um einen Vektor. Dieser enthält die Punkte des Bildes, die ihn charakterisieren. Er kann somit einen Punkt, eine Kurve oder eine Region darstellen. Es hat sich dabei aus vielen Gründen bewährt lokale Features globalen vorzuziehen, u. a. da sie um ein vielfaches schneller zu berechnen sind [ref]. Lokale Verfahren haben jedoch den Nachteil, dass sie oft mehr Speicher benötigen: Ein Bild kann eine beliebig große Anzahl Features enthalten. Aus diesem Grund ist es notwendig die einzelnen Feature Vektoren kompakt darzustellen. Eine Reduzierung der Dimensionen der Feature Vektoren kann durch verschiedene Verfahren erreicht werden, siehe Kapitel [REF].

Ein Feature eines Bildes kann unterschiedlichen Typs sein.

Der Bereich Feature Detektoren beschäftigt sich zunächst mit verschiedenen Algorithmen zum Auffinden von \textit{interest points}, den Punkten im Bild die ein Feature bilden können. 
Durch die Feature Extraktion werden aus den \textit{intereset points} die Deskriptoren des Bildes abgeleitet. Auf Basis dieser Deskriptoren kann ein effizienter Vergleich der Bilder umgesetzt.

\begin{enumerate}	
	\item Globale vs Lokale Features (robust, effizient, eindeutig, viele)
	\item Lokal: Besser für Skalierungen, Rotationen, Verdeckungen.
	\item Repräsentation von Image Features (Binary Object?)	
\end{enumerate}

\section{Feature Detektion}

Im Bereich Computer Vision zielt die Feature Detektion darauf ab Muster innerhalb von Bildern zu entdecken. Diese Muster werden durch das lokale Vergleichen von Strukturen, Pixeln und deren Nachbarschaft, effizient ermittelt. Diese entdeckten Muster bilden die Features eines Bildes. Ein Feature Detektor sollte einer Reihe von Eigenschaften genügen um praktisch einsetzbar zu sein. Je nach Anwendung fällt den Eigenschaften unterschiedlich viel Gewicht zu:

\begin{enumerate}
	\item \textbf{Robustheit} Die Features sollen auch unter Rotation, Skalierung, Translation und Rauschen identifizierbar sein.
	\item \textbf{Genauigkeit} Der Algorithmus soll die Positionen der Features präzise bestimmen. Notwendig wenn beispielsweise exakte Zusammenhänge berechnet werden sollen.
	\item \textbf{Allgemeinheit} Bestimmt die generelle Einsetzbarkeit eines Algorithmus. Ein Feature Detektor für medizinische Bilder spezieller Annahmen treffen, als einer für eine allgemeine Online Suche.
	\item \textbf{Effizienz} Die Verarbeitung sollte in Echtzeit erfolgen.
	\item \textbf{Wiederholbarkeit} Dieselben Features eines Bildes sollten in wiederholten Läufen erkannt werden.
\end{enumerate}

\subsection{Harris-Operator}

https://courses.cs.washington.edu/courses/cse455/09wi/Lects/lect6.pdf

\subsection{Scale Invariant Point Detection}

TODO: DoG für SIFT

https://courses.cs.washington.edu/courses/cse455/09wi/Lects/lect6.pdf

\section{Feature Extraktion}

Die Feature Extraktion zielt darauf ab die \textit{intereset point} kompakt darzustellen um sie mit anderen \textit{interest points} zu vergleichen (BSP). Mathematisch handelt es sich um eine Form der Reduzierung von Dimensionen. Dies ist gewünscht, da die unverarbeiteten Daten meist zu groß sind als das ein Algorithmus sie in annehmbarer Zeit verarbeiten könnte. Diese Verfahren sind am Erfolgreichsten, wenn in den Daten nicht viele Informationen bzw. viele Redundanzen vorhanden sind. Die resultierende Darstellung ist der Deskriptor. Dieser enthält den Informationen um den Pixel und seine Nachbarschaft unter Berücksichtigung der Orientierung und Skalierung.

CITE "A local descriptor a compact representation of a point’s local neighborhood. In contrast to global descriptors describing a complete object or point cloud, local descriptors try to resemble shape and appearance only in a local neighborhood around a point and thus are very suitable for representing it in terms of matching." (Dirk Holz et al.).

Im Folgenden werden die Deskriptoren SIFT, LBP und GLOH vorgestellt. Diese Deskriptoren bieten alle eine Kodierung die Rotationen und teilweise Skalierungen berücksichtigen und sich somit prinzipiell für den Einsatz eignen.

\cite{ifd2016}

\subsection{Local Binary Patterns}

Der Local Binary Patterns (LBP) Deskriptor vergleicht jeden Pixel eines Bildes wird mit seiner direkten Nachbarschaft um lokale Beschreibungen zu erzeugen. Auf diese Weise wird für jeden Pixel eine $3 \times 3$ Matrix betrachtet. Für jeden Pixel der Nachbarschaft wird überprüft, ob die Intensität über einem vorgegebenen Schwellwert liegt. Falls dies der Fall ist wird in die resultierende Matrix eine 1, andernfalls eine 0 eingetragen. Zeilenweise konkateniert ergeben diese Matrizen einen acht stelligen Binärstring (der betrachtete Pixel selbst wird nicht mitgezählt).

http://bytefish.de/blog/local\_binary\_patterns/

\subsection{GIST}

http://cvcl.mit.edu/Papers/IJCV01-Oliva-Torralba.pdf

\subsection{Scale-invariant feature transform}

SIFT ist sowohl ein Feature Detektor als auch Deskriptor der erstmals 199x von Lowe eingeführt wurde. Der scale-invariant feature transform (SIFT) Detektor findet eine Menge \textit{interest points} durch den Difference of Gaussian (DOG) Operator. 

\subsubsection{Detektor}

Zunächst werden aus dem vorliegen Bild $I(x, y)$ zufällige Patches $P$ entnommen. Innerhalb dieser Ausschnitte werden dann lokale keypoints durch den Difference of Gaussian Operator berechnet. Dieser subtrahiert eine verzerrte Version eines Originalbildes von einer weniger verzerrten Version des selben. Für Graustufenbilder wird dies durch eine Konvolution gausscher Kernel mit verschiedenen Standardabweichungen realisiert. Dadurch das nur räumliche Informationen höherer Bereiche unterdrückt werden, fungiert dies als Band Pass Filter und hebt so die Sichtbarkeit von Kanten hervor.

Bei dem Aufbau des Feature Vektoren pro \textit{interest point} wird die lokale Orientierung abgeschätzt. Auf diese Weise sind die SIFT Deskriptoren invariant gegenüber Rotationen. Der SIFT Algorithmus berechnet ein Histogramm der Orientierung der Gradienten. Hierfür werden zufällig Punkte aus der Nachbarschaft ausgewählt. Der Extremwert des Histogramms wird hier als dominante Orientierung verwendet.

\subsubsection{Deskriptor}

Für jeden durch den Detektor gefundenen keypoint wird nun ein Featurevektor gebildet. Der Featurevektor enthält Informationen über die Nachbarschaft in Form der Gradienten eines jeden Punktes in der Nachbarschaft. Das Fenster für die Auswahl der Nachbarschaft wird auf dem keypoint zentriert und in vier Teilfenster unterteilt. Die Gradienten in allen Teilfenster werden in acht Richtungen quantisiert, sodass der resultierende Deskriptor 128 Dimensionen enthält.

PRO

\begin{enumerate}
	\item Änderungen im Grenzwert von Position und Orientierung verändern den Feature Vektor kaum.
	\item Erreicht eine kompakte Darstellung. Robustheit ist in praktischen Anwendungen enorm hoch, auch wenn der Algorithmus keine affine Invarianz bietet.
	\item In Vergleichen wurden sehr gute Ergebnisse gegen andere Algorithmen erzielt.
\end{enumerate}

CON

\begin{enumerate}
	\item Die Konstruktion des SIFT Deskriptors ist aufwendig
	\item Hohe Dimensionalität des Feature Vektors, was zu einer langen Berechnungszeit führt.
\end{enumerate}

\subsection{Distanzmetriken}

TODO Zitat hier \cite{mmd2011}

\section{Dimensionality Reduction}

Für die weitere Verarbeitung sollen nur die Features ausgewählt werden. Da schon pro Bild eine große Menge an Feature Vektoren erzeugt wird, gilt es nur die wichtigsten herauszufiltern. 

\subsection{SIFT-PCA}

EXTENSION: PCA-SIFT

\begin{enumerate}
	\item reduce the high dimensionality of original SIFT descriptor using the
standard Principal Components Analysis (PCA)
	\item extracts a 41 * 41 patch at the given
scale and computes its image gradients in the vertical and horizontal directions and
creates a feature vector from concatenating the gradients in both directions
	\item feature vector is of length 2 * 39 * 39 = 3042 dimensions
	\item gradient image vector is projected into a pre-computed feature space, resulting a feature vector of length 36 elements
	\item vector is then normalized to unit magnitude to reduce
the effects of illumination changes
\end{enumerate}

\subsection{Autoencoder}

Autoencoder sind eine Art neuronales Netzwerk und werden für unbeaufsichtigtes lernen und Kompression verwendet. Zunächst wird hierfür ein Überblick über neuronale Netze gegeben und dann die Funktionsweise eines Autoencoders erläutert. Im Weiteren werden spezielle Varianten des Autoencoders vorgestellt, die zur Optimierung des Ergebnisses dienen.

\subsubsection{Neuronale Netze}

Künstliche neuronale Netzwerke (ANN, NN) werden seit den 50er Jahren erforscht. Durch die wachsende Rechenleistung und neue Forschungsgebiete wie Deep Learning und Machine Learning finden NN seit Anfang 2000 vermehrt praktische Anwendung und akademische Zuwendung. NN sind dem Aufbau und der Funktionsweise des menschlichen Gehirns nachempfunden. Dadurch das NN von Natur aus hoch parallel arbeiten, eignen sie sich vor allem für parallele Architekturen und die Verarbeitung großer Datenmengen. 

Ein NN $ TODO $ besteht aus einer Menge Neuronen $ TODO $ die in Schichten im Netzwerk angeordnet sind. Ein Neuron $n_i$ besitzt einen Aktivierungszustand $z_i$.  Neuronen benachbarter Schichten sind durch eine Gewichtsmatrix $d$ "komplett" miteinander verbunden. Solch ein Netzwerk verarbeitet ein Signal, welches hier als Vektor $x \in [0,1]^n$ dargestellt wird. Die erste Schicht des Netzwerks, der Input Layer, leitet das Signal nur an die nächste Schicht weiter. Die letzte Schicht, der Hidden Layer, dient zur Ausgabe des Ergebnisvektors $z \in [0,1]^n$. Zwischen diesen beiden Schichten können sich beliebig viele Hidden Layer befinden. Sobald ein Neuron in einem HiddenLayer ein Signal erreicht, wird überprüft ob das Neuron aktiviert wird. Die Überprüfung erfolgt durch die Aktivierungsfunktion $s$. Häufig wird hier die sigmoid Funktion $s(x) = \frac{1}{1+e^-x}$ verwendet. Wird ein Neuron aktiviert, so wird das resultierende Signal durch die Ausgabefunktion $out$ berechnet und an alle Neuronen in der folgenden Schicht weitergeleitet. 

\subsubsection{Funktionsweise}
Ein Autoencoder (AE) ist ein spezielles neuronales Netzwerk, dass die Identitätsfunktion lernen soll. AE dienen der Reduzierung der Dimensionen eines Features und können unbeaufsichtigt lernen. Um das Original als Ergebnis erhalten zu können, muss die Anzahl der Neuronen des \textit{Inputlayers} der Anzahl der Neuronen im \textit{Outputlayer} entsprechen. Die Anzahl der Neuronen im \textit{Hiddenlayer} ist geringer, um die komprimierte Darstellung des Features zu erreichen. Werden mehrere \textit{Hiddenlayer} verwendet, so nimmt die Neuronenanzahl von Layer zu Layer ab um die Anzahl der Dimensionen weiter zu verringern. Dieser Vorgang ist die Enkodierung und liefert die gewünschte komprimierte Abbildung. Die Dekodierung ist umgekehrt aufgebaut, um das Original aus den komprimierten Feature Ebene für Ebene zu rekonstruieren. Wie gut die Dekodierung gelungen ist, lässt sich dann anhand eines Vergleichs der Distanz des Original und der Rekonstruktion bewerten. Formal wird ein Eingabevektor $x \in [0,1]^n$ auf einen Vektor $y \in [0,1]^p$ durch $y = encode_{W,b}(x) = s(Wx + b)$ abgebildet. $W$ ist die Gewichtsmatrix $n \times p$ und $b$ der Bias-Vektor. Dies sind die Parameter, welche durch den AE optimiert werden sollen. Die Rekonstruktion erfolgt durch die Dekodierungsfunktion:$z \in [0, 1]^n$ wird dann durch $z = decode_{W', b'}(y) = s(W'y + b')$.

\begin{enumerate}
	\item Vorteile eines Autoencoders gegenüber anderen Verfahren (supervised, pretraining)
	\item Grafik 1 + 5 Layer (Encode / Decode)
	\item Backpropagation / Lernregel (Gradientenverfahren / Fehlerfunktion)
\end{enumerate}

\cite{ssn1997}

\subsubsection{Stacked Autoencoder}

\subsubsection{Denoising Autoencoder}

\section{GPGPU Programmierung}

General Programming on Graphics Processing Units (GPGPU) ist ein Verfahren um die massive Parallelität von Grafikkarten zur Berechnung allgemeiner mathematischer Probleme zu nutzen. Um die Parallelität der Grafikkarte effizient Nutzen zu können, muss das Problem als Matrix vorliegen. Da Alle Kerne eine GPU pro Takt dieselbe Operation ausführen, muss auf Verzweigung möglichst verzichtet werden: Je mehr Kerne  in einem Takt keine Operation ausführen, desto mehr nährt sich die Ausführungsgeschwindigkeit einer sequentiellen Ausführung an.

\subsection{Nvidia cuda}

Ein Programm das auf einer Nvidia Grafikkarte ausgeführt werden soll muss in der cuda Sprache geschrieben sein. Hierbei handelt es sich um eine Erweiterung von C um primitive und Funktionen für Berechnungen auf der Grafikkarte. Die Hauptfunktion wird als Kernel bezeichnet und wird durch das Schlüsselwort \textit{\textunderscore\textunderscore kernel\textunderscore\textunderscore} identifiziert. Zunächst wird notwendige Speicher auf der Grafikkarte allokiert. Nachdem die Daten von Host zur Grafikkarte (Device) kopiert wurden, kann die Berechnung gestartet werden. Nach Durchführung oder zwischen Berechnungen kann dann das Ergebnis zurück zum Host kopiert werden. Diese Operationen weisen, vor allem bei großen Datenmengen, eine nicht unbeachtliche Latenz auf. Folglich sollte das Kopieren von Daten möglichst nur selten erfolgen.

\begin{enumerate}
	\item Grids, Blocks, Threads
	\item Optimierungen (?)
	\item cuda Beispiel
\end{enumerate}

\subsection{Neuronale Netze auf Grafikkarten}

TODO
\chapter{Analyse}

Ziel der Arbeit ist es, dass ein Anwender die in der HsH Datenbank vorhandenen Bilddaten in eine Menge von k verschiedenen Gruppen einteilen kann, um so semantische Informationen über die Bilder zu gewinnen. Das k kann hierbei vom Anwender vorgegeben werden um verschiedene Gruppen zu erhalten und so unterschiedliche Zusammenhänge zwischen den Bildern zu entdecken. Bei einer Teilmenge der vorhandenen Bilddaten handelt es sich beispielsweise um medizinische Abbildungen wie Röntgenaufnahmen. Wenn Unregelmäßigkeiten eines Organs bei der Aufnahme eines Patienten mit existierenden Aufnahmen verglichen werden sollen, ist eine textuelle Suche denkbar. Hierfür ist eine Speicherung von Suchbegriffen zu einem Bild notwendig. Da aber bereits ein großer Bestand an nicht kategorisierten Bilddaten vorhanden ist, wäre dies aufwendig manuell zu bestätigen. Weiterhin ist es mit einer textbasierten Suche nicht möglich Details eines Bildes an sich zu vergleichen. Auf diese Weise würde nicht nur nach einer Menge von Organen gesucht werden, sondern nach Übereinstimmungen von Mustern auf dem Bild.

\begin{enumerate}
	\item Ziel ist die Kategorisierung von Bilddaten
	\item Große Datenbank der Fakultät, "Bildersammlung"
	\item Größe der Gruppen variieren, um verschiedene Labels zu gewinnen
	\item Verfahren aus der Literatur um Informationen aus Bildern zu gewinnen, die diese beschreiben (Deskriptoren) \begin{enumerate}
		\item LBP http://bytefish.de/blog/local\_binary\_patterns/
		\item GIST http://cvcl.mit.edu/Papers/IJCV01-Oliva-Torralba.pdf
		\item SIFT 
	\end{enumerate}
	\item Ansätze zur Klassifizierung von Bildern anhand Deskriptoren
		\item Bag of Visual Words (Literatur, ...)
		\item Autoencoder
	\item Ausführung auf GPU
		\item Bestandteile Bag of Visual Words \begin{enumerate}
			\item K-Means
			\item Histogramm
		\end{enumerate}
		\item Neuronale Netze, Bibliotheken 
\end{enumerate}

\section{Anforderungen} 

\begin{enumerate}
	\item Features sollen scale invariant sein
	\item Ausführung auf GPU
\end{enumerate}

\section{Feature Deskriptoren}

\begin{enumerate}
	\item verschiedene Ansätze aus der Literatur benennen, diskutieren

\end{enumerate}

\section{Verfahren zur Klassifizierung}

\begin{enumerate}
	\item BoVW
	\item ML: SIFT AE
\end{enumerate}
\chapter{Konzept}

Das Kapitel Konzeption beschäftigt sich zunächst mit dem Prozess der Feature Extraktion. Hierfür wird der SIFT Algorithmus nach Lowe genutzt. Folgend wird das Bag of Visual Words Modell näher betrachtet: Es werden auf Basis der Analyse parallele Varianten des Clustering und Histogramm Algorithmus entworfen, die sich zur Ausführung auf SPMD Architekturen eignen. Anschließend wird der geplante Ablauf zum Generieren eines Bag of Visual Word Modells und Labelling eines Modells durch das Modell erläutert.
Abschließend wird auf der Basis der Arbeit von [REF] ein Autoencoder eingeführt, der aus SIFT \textit{keypoints} eine Darstellung des Features in nur 36 Dimensionen lernt.

\section{Feature Extraktion}

Die Extraktion der Features ist die Basis für beide Varianten der Klassifizierung. Das Bag of Visual Words Modell nutzt die von SIFT erzeugten Feature Deskriptor für die weitere Verarbeitung. Der Autoencoder hingegen arbeitet mit Gradienten der \textit{keypoints} die vom SIFT Detektor ermittelt wurden. Aus diesem Grund werden die Feature-Vektoren und \textit{keypoints} beide berechnet und getrennt gespeichert.
Der SIFT Deskriptor enthält 128 Dimensionen und ist so für einen Vergleich nur schwer geeignet, da pro Bild ca. 100 bis 1000 Feature Vektoren generiert werden. Es werden daher im folgenden zwei Ansätze vorgestellt, die die Dimensionalität der Features reduzieren und eine durch Grafikkarten gestützte Berechnung von ähnlichen Bildern ermöglichen.

\todo{Was noch hier, oder überhaupt?}

\section{Ansatz 1: Bag of Visual Words}

In der Analyse wurde bereits sequentielle Varianten des Lloyd und Histogramm Algorithmus vorgestellt und aufgezeigt, an welchen Stellen eine Parallelisierung der Berechnung durch Grafikkarten erfolgen kann. Im Folgenden wird aus diesen Informationen je Algorithmus eine parallele Version für SPMD Prozessoren abgeleitet.
In den beiden nachfolgenden Abschnitten Generierung des Modells und Labeling eines Bildes wird auf den Programmaufbau und -ablauf näher eingegangen. Es werden die wesentlichen Funktionen, ihre Parameter und Aufrufe skizziert.

\subsection{Parellisierung von Llyods Algorithmus}

Der Thread in einem Block mit der ID 0 fungiert hier als Master für die anderen Threads. Die Initialisierung der Cluster mit zufälligen Vektoren aus $v$ wird ebenfalls von diesem übernommen. Die Zuweisung von Vektoren zu Clustern nimmt $\Theta(nk)$ Zeit in Anspruch, wobei $n$ die Anzahl Vektoren und $k$ die Anzahl der Cluster ist. Diese Phase kann parallelisiert werden, in dem pro Feature Vektor ein Thread verwendet wird: Jeder Thread berechnet für seinen Feature Vektor die Distanz zu allen Clusterschwerpunkten und bestimmt den Index des Clusters, der am Nächsten ist. Dieser Prozess ist in Pseudocode in Zeile 6 bis 8 ausgedrückt. Bevor die Cluster aktualisiert werden, müssen die Threads synchronisiert werden: Andernfalls ist nicht garantiert, dass die Berechnung jedes Threads abgeschlossen ist.

\lstset{language=C}
\begin{lstlisting}[mathescape=true]
kmeans_gpu
	if threadId == 0
		$c_{j} = rand(p_{i}) \in P, \: j = 1,...,k, \: c_{j} \neq c_{i} \: \forall i \neq j$
	synchronize threads
	until convergence
		for each $x_{i} \in P_{threadId}$
			$l_{i} = argminD(c_{j}, p_{i})$
		synchronize threads
		if threadId == 0
			for each $p_{i} \in P$
				$c_{l_{i}} = c_{l_{i}} + p_{i}$
				$m_{l_{i}} = m_{l_{i}} + 1$
			for each $c_{j} \in C$
				$c_{j} = \frac{1}{m_{j}} c_{i}$
\end{lstlisting}

\subsection{Parallele Reduzierung von Histogrammen}

\todo{Allgemeines parallel reduction Prinzip in SIMD (Pseudocode?)}

Die Berechnung eines Histogramms kann hervorragend parallelisiert werden, da die Operation assoziativ und kommutativ ist: Es spielt keine Rolle in welcher Reihenfolge die Daten abgearbeitet werden bzw. in welcher Reihenfolge die Klassen inkrementiert werden. Wenn das zu beschreibende Histogramm im global Speicher vorliegt, wird die Berechnungsgeschwindigkeit stark reduziert, da viele Threads auf die gleichen Speicheradressen des Histogramms schreibend zugreifen. Damit es nicht zu Lese- / Schreibanomalien kommt, muss das Inkrementieren einer Klasse atomar sein, d.h. zwischen Lese- und Schreibzugriff darf kein anderer Thread auf die Adresse zugreifen. Dies wird in CUDA durch die Operation \textit{atomicAdd} realisiert. Damit die Anzahl an Threads die auf dieselbe Adresse schreiben eingeschränkt wird, arbeitet jeder Block auf einem lokalen Histogramm im \textit{shared memory}. Wenn alle Blöcke ihre lokalen Histogramme berechnet haben, müssen diese noch in das Histogramm im \textit{global memory} kumuliert werden.

\lstset{language=C}
\begin{lstlisting}
__global__
void histogram_kernel (float *buffer, long size, int *histo, int bins) {
	extern __shared__ int *copy[];
	
	if (threadIdx.x < bins) {
		copy[threadIdx.x] = 0;		
	}
	__syncthreads();

	int id = threadIdx.x + blockDim.x * gridDim.x;
	int stride = blockDim.x * gridDim.x;
	
	while (i < stride) {
		int bin = buffer[i] / bins; 
		atomicAdd(&(copy[bin]), 1);
		i += stride;	
	}
	__syncthreads();
	
	if (threadIdx.x < bins) {
		atomicAdd(&(histo[threadIdx.x]), copy[threadIdx.x]);		
	}
}
\end{lstlisting} 

\subsection{Aufbau des Bag of Visual Words Algorithmus}

Das Bag of Visual Words Modell soll zwei Anwendungsfälle unterstützen. Zunächst muss aus einer Menge von Bildern ein Modell generiert werden. Da verschiedene Modelle erstellt werden sollen, müssen diese gespeichert und auch wieder eingelesen werden können. Der Abschnitt Generierung des Modells beschäftigt sich mit einem Entwurf solch eines Systems. Sofern ein Modell generiert wurde, soll es einem Anwender möglich sein ein neues Bild anhand des Modells zu labeln. Dieser Prozess ist in Kapitel Labeling eines Bildes dargestellt.

\subsubsection{Generierung des Modells}

Die Generierung eines Modells kann durch die Funktion \textit{generateModel} gestartet werden. Als Parameter werden der Pfad für die Bilddaten \textit{imageDir}, der Zielpfad \textit{modelPath} und die Anzahl der Cluster \textit{k} erwartet. Der Ablauf der folgenden Funktionsaufrufe ist in Abbildung \ref{img:concept_bovw_1} dargestellt. Im ersten Schritt wird \textit{extractFeatures} aufgerufen um alle SIFT-Features der Bilder, die in \textit{imageDir} enthalten sind, zu extrahieren. Als nächstes werden durch \textit{clusterFeatures} die Features in \textit{k} Cluster gruppiert. Die Berechnung der Cluster, der Distanzen von Features zu Clustern und des Konvergenzkriteriums erfolgt durch die GPU. Als Ergebnis werden die $k$ berechneten Schwerpunkte der Cluster und die Mitgliedschaft der Features zurückgegeben. Abschließend speichert \textit{saveModel} die Cluster unter \textit{$<modelPath>/clusters$} und die Mitgliedschaft unter \textit{$<modelpath>/membership$}. \todo{single extractFeature}

\begin{figure}
	\centering
	\includegraphics[scale=0.8]{images/concept_bovw_1.png}
	\caption{Funktionen zur Generierung eines Modells}
	\label{img:concept_bovw_1}
\end{figure}

\subsubsection{Labeling eines Bildes}

Sofern ein Modell erstellt wurde, können auf dessen Basis Bilder verglichen werden. In Abbildung \ref{img:concept_bovw_2} sind schematisch die aufeinanderfolgenden Funktionsaufrufe dargestellt. Die Funktion \textit{getImageLabels} wird mit dem Pfad des Modells und des zu vergleichenden Bildes aufgerufen. Das Modell wird durch \textit{readModel} eingelesen und die Clusterschwerpunkte initialisiert. Die SIFT Features werden, wie bei der Generierung, durch \textit{extractFeatures} ermittelt. Die Funktion \textit{selectLabels} berechnet durch \textit{computeFrequencies} das Histogramm der Visual Words aus den Cluster und Features auf der GPU. Auf dieser Basis werden dann die top X Labels ermittelt und zurückgegeben.

\begin{figure}
	\centering
	\includegraphics[scale=0.8]{images/concept_bovw_2.png}
	\caption{Funktionen zur Gewinnung von Labels eines Bildes}
	\label{img:concept_bovw_2}
\end{figure}

\section{Ansatz 2: Autoencoder}

In diesem Ansatz wird zur Reduzierung der Dimensionen der Feature-Vektoren ein Stacked Denoising Autoencoder verwendet wie er in der Arbeit von Zhao \cite{aed2016} vorgeschlagen wurde. Der Autoencoder soll eine komprimierte Darstellung der Gradientenvektoren erzielen, die aus den \textit{intereset points} berechnet werden. Da dieser Vektor 3042 Werte enthält, besitzt der Autoencoder in der Eingabeschicht 3042 Neuronen. Der Encoder des vorgeschlagenen Modells besteht aus fünf Schichten, deren Neuronenanzahl sukzessive reduziert wird, bis schließlich eine Darstellung in 36 Dimensionen erreicht wird. Abbildung \ref{img:ae_model} zeigt die Schichten des Autoencoders. In der Arbeit wurde bereits aufgezeigt, dass der modellierte Autoencoder \textit{state of the art} Ergebnisse erzielt: Die Ergebnisse des Autoencoders wurden unter verschiedenen Kriterien mit den Ergebnissen der SIFT-PCA und TODO Methode verglichen. Dabei erkannte der Autoencoder in fast allen vielen die gleichen Features, jedoch durch einen 36 statt 128 dimensionalen Feature-Vektor. Theoretisch wird hier also das gleiche Ergebnis in einem Drittel der Zeit ermittelt.

\begin{figure}
	\centering
	\includegraphics[scale=0.6]{images/ae_model.png}
	\caption{Schichten des verwendeten Autoencoders \cite{aed2016}}
	\label{img:ae_model}
\end{figure}

\todo{Mit Referenz auf die Arbeit ist es plausibel dieses Modell zu verwenden?}

\todo{In der Konzeption bereits TensorFlow erwähnen und somit Beschleunigung durch cuda Bindings?}
\chapter{Implementierung}

TODO

\section{GPU Kernel des neuronalen Netzes}

TODO
\chapter{Evaluierung}

Die Qualität der Ergebnisse des in der Implementierung realisierten Modells wird in diesem Kapitel durch ein Testexperiment überprüft. Weiterhin werden Ausführungszeiten für beispielsweise verschieden große $k$ beim Bag of Visual Words betrachtet, um eine Einsicht in die Performance der Algorithmen zu gewinnen.
Im Kapitel Experimentaufbau wird zunächst das Experiement sowie verwendete Metriken und Ergebnistypen behandelt. Nachfolgend wird erläutert, wie die großen Menge an benötigten Trainings- und Testdaten wiederholbar und automatisiert aufgebaut wird. Dieser Abschnitt illustriert auch, wie eine Durchführung des Experiments vonstattengeht. Im letzten Teil werden dann konkrete Testgruppen aus den Caltech101 \cite{cal2004} Bilddaten erzeugt. Diese Menge an Bilddaten hat große Verbreitung im Bereich der Objekterkennung gefunden. Auf diese Weise ist ein Vergleich mit Arbeiten Anderer prinzipiell möglich.

\section{Testdaten und Testgenerierung}

Der Abschnitt Testdaten stellt die hier verwendete Menge von Bildern vor, die Caltech101, welche extra für den Test von Algorithmen bezüglich der Objekterkennung in Bildern entwickelt wurde.\newline
Im Folgenden Abschnitt zur Testgenerierung wird ein Verfahren zur zufälligen Auswahl von Trainings- und Testbildern unter Berücksichtigung verschiedener Restriktionen, wie z.B. dem Verhältnis der Anzahl von Trainings- zu Testbildern vorgestellt.

\subsection{Testdaten}

Als Testmenge wurden die Bilder der Caltech101 Menge verwendet. Bei Caltech101 handelt es sich um eine weit verbreitete Menge von Bilddaten, die vorwiegend zum Test von Algorithmen bezüglich der Objekterkennung in Bildern dient. Insgesamt liegen, wie der Name sagt, 101 Kategorien vor, die jeweils zwischen 40 und 800 Bildern enthalten. Auf der offiziellen Webseite \footnote{http://www.vision.caltech.edu/Image\textunderscore Datasets/Caltech101/} und im Artikel der Autoren wir empfohlen, die eigene Arbeit mit derer anderen vergleichbar zu halten, indem:

\begin{itemize}
	\item Eine feste Anzahl an Trainings- und Testbildern verwendet wird.
	\item Experimente mit einer zufälligen Auswahl an Bildern wiederholt werden.
	\item Ähnlich viele Bilder, wie in den Arbeiten anderer, verwendet werden (1, 3, 5, 10, 15, 20 oder 30 Trainingsbilder; 20 oder 30 Testbilder).
\end{itemize}

Die Caltech101 Daten liegen nach Download kategorisiert im JPG-Format vor. Die Struktur wurde so beibehalten und ist noch für die Testgenerierung relevant. Direkt unter dem Caltech101-Ordner ist pro Kategorie ein Ordner vorhanden, der die jeweiligen Bilder immer im gleichen Namensschema enthält:\newline

\dirtree{%src
.1 Caltech101.
.2 accordion.
.3 image\textunderscore 0001.jpg.
.3 image\textunderscore 0002.jpg.
.3 ....
.2 airplanes.
.3 ....
.2 ....
}

Abbildung \ref{img:strawberries} zeigt vier Bilder aus der Kategorie \enquote{Erdbeere}. Neben Bildern von realen Rosengewächsen sind auch Zeichnungen und Objekte enthalten, die Form und Farbe der Erdbeere nachempfunden sind. Auch sind die Objekte auf einem Bild zum Teil in unterschiedlicher Menge vorhanden. Durch dies Variation ist ein Algorithmus so gefordert, tatsächlich eine Abstraktion zu lernen.

\begin{figure}
	\centering
	\includegraphics[scale=0.5]{images/strawberry.png}
	\caption{Verschiedene Bilder aus der Kategorie \enquote{Erdbeere} der Caltech101 Bilddaten.}
	\label{img:strawberries}
\end{figure}

\subsection{Testgenerierung}

Die Generierung von Testdaten bietet sich aus mehreren Gründen an. Zum einen sind enorm viele Trainingsdaten notwendig, um ein leistungsfähiges Modell zu generieren, zum anderen sollen im Test ca. 2000 Bildpaare verwendet werden. Ein manueller Testaufbau wäre fehleranfällig und nicht sehr flexibel. Da ein praktisch taugliches Modell erst durch die Variation einiger Parameter gefunden werden kann, ist es wünschenswert, Testdaten mit verschiedenen Eigenschaften generieren zu können:

\begin{itemize}
	\item Die Anzahl der Kategorien sollte bestimmbar sein. Dies entspricht einer Kategorie der Caltech101-Daten. Somit sind hier theoretisch bis zu 101 Kategorien im Test denkbar.
	\item Das Verhältnis bzw. Die Anzahl an Trainings- und Testbildern muss definierbar sein.
\end{itemize}

Letztendlich sollen die Histogramme der \textit{Visual Words} zweier Bilder miteinander verglichen und so die Ähnlichkeit gemessen werden. Ein Programm automatisiert daher die Generierung solcher Paare: Es werden zufällige Bildpaare aus ausgewählten Kategorien (\textit{airplanes}, \textit{anchor}, ...) selektiert. Diese Paare, sowie die Information, ob die Bilder in der selben oder einer verschiedenen Kategorie liegen, stellen einen Testkandidaten dar. Zwei Bilder liegen dabei in der selben Klasse, wenn sie im selben Ordner im Dateisystem, also hier im Caltech101-Ordner, enthalten sind. Das Ergebnis wird dann als Datei \textit{test \textunderscore time.txt} gespeichert. Die Pfade der Bilder werden hierbei relativ zum Caltech101-Ordner gespeichert, die Information über die Kategorie wird als \enquote{+} bzw. \enquote{-} kodiert. Eine Datei für die Kategorien \textit{airplanes} und \textit{anchor} könnte dann so beginnen:\newline

\begin{lstlisting}
airplanes/image_0023.jpg airplanes/image_0009.jpg +
airplanes/image_0002.jpg anchor/image_0015.jpg -
anchor/image_0013.jpg airplanes/image_0002.jpg -
anchor/image_0001.jpg anchor/image_0005.jpg +
airplanes/image_0006.jpg anchor/image_0006.jpg -
...
\end{lstlisting}

Neben den zu verwendenden Kategorien muss bei Erzeugung die Anzahl der Testkandidaten, das Verhältnis von positiven zu negativen Kategorien sowie die Anzahl der Trainings- und Testbilder angegeben werden. \newline
Die Bilder, welche durch das Programm für das Training ausgewählt wurden, werden separat als \textit{train\textunderscore time.txt} gespeichert. Pro Zeile ist hier der relative Pfad des Bildes innerhalb des Caltech101-Ordners enthalten.

\section{Experimentaufbau}

Das Experiment soll sowohl die Feature-Kompression durch einen Autoencoder testen als auch die Kategorisierung bzw. den Vergleich der Bilder durch den Bag of Visual Words. Aus diesem Grund ist das Experiment zweigeteilt: 

\begin{enumerate}[(a)]%
	\item In dieser Variante findet ein reiner Test des Bag of Visual Words statt. Hierfür werden durch SIFT die Feature-Deskriptoren von Trainingsbildern extrahiert und direkt als Eingabe an den Bag of Visual Words gegeben. Anschließend folgt die Verarbeitung der Testbilder.
	\item Hier wird nach Extraktion der \textit{keypoints} durch den SIFT-Detektor der Feature-Deskriptor durch den Autoencoder erzeugt. Die so erhaltenen Features werden dann wie in (a) durch den Bag of Visual Words gruppiert und anschließend die \textit{Visual Words} der Testbilder erzeugt.
\end{enumerate}

Wichtig ist, dass pro Durchführung des Experiments in beiden Varianten die gleichen Trainings- und Testbilder verwendet werden, damit die Deskriptoren miteinander vergleichbar sind. \newline
Nach Erzeugung des Modells mit den Trainingsbildern, werden nun die Features der Testkandidaten extrahiert und pro Bild dies \textit{Visual Words} berechnet. Die Ähnlichkeit \textit{s (similarity)} der resultierenden Histogramme $h_1$ und $h_2$ wird dann als $1 - $ \textit{MSE (mean squared error)} gemessen:

$$s(h_1, h_2) = 1 - MSE(h_1, h_2)$$
$$MSE(h_1, h_2) = \frac{1}{n}\sum_{i=0}^{n}(h_{1_i} - h_{2_i})^{2}$$

Hier gibt es zwei Fälle zu unterscheiden:

\begin{itemize}
	\item \textbf{True Positives} Bei \textit{True Positives} handelt es sich um zwei Bildern die entweder in der gleichen oder einer verschiedenen Klasse liegen und die Vorhersage des Modells diesbezüglich korrekt ist.
	\item \textbf{False Positives} In diesem Fall ist die Klassifizierung durch das Modell nicht korrekt: Bei gleicher Klasse wurde eine geringe Ähnlichkeit erkannt, bei verschiedenen eine Hohe.
\end{itemize}

Damit ein Modell zuverlässige Ergebnisse liefert, muss es größtenteils \textit{True Positives} erkennen, bzw. der Anteil der \textit{True Positives} sollte im Verhältnis zu den \textit{False Positives} bei weitem überwiegen. Für eine visuelle Darstellung dieses Verhältnisses eignet sich die \textit{Receiver Operating Characteristic (ROC)}: Diese stellt für verschiedene Parameter die \textit{True Positives} auf der Ordinate und die \textit{False-Positives} auf der Abzisse dar, sodass der optimale Parameter anhand der resultierenden Kurve abgelesen werden kann.

\section{Experimentdurchführung}

\todo{LALALA}

\bibliographystyle{alpha}
\bibliography{sources}

\end{document}


