\chapter{Analyse}

Ziel der Arbeit ist es die in der HsH Datenbank vorhandenen Bilddaten in eine Menge von k verschiedenen Gruppen einzuteilen, um so semantische Informationen über die Bilder zu gewinnen. Das k kann hierbei vom Anwender vorgegeben werden um verschiedene Gruppen zu erhalten und so unterschiedliche Zusammenhänge zwischen den Bildern zu entdecken. Bei einer Teilmenge der vorhandenen Bilddaten handelt es sich beispielsweise um medizinische Abbildungen wie Röntgenaufnahmen. Wenn Unregelmäßigkeiten eines Organs bei der Aufnahme eines Patienten mit existierenden Aufnahmen verglichen werden sollen, ist eine textuelle Suche denkbar. Hierfür ist eine Speicherung von Suchbegriffen zu einem Bild notwendig. Da aber bereits ein großer Bestand an nicht kategorisierten Bilddaten vorhanden ist, wäre dies aufwendig manuell zu bestätigen. Weiterhin ist es mit einer textbasierten Suche nicht möglich Details eines Bildes an sich zu vergleichen. Auf diese Weise würde nicht nur nach einer Menge von Organen gesucht werden, sondern nach Übereinstimmungen von Mustern auf dem Bild.

1. Anforderungen 
	- Features sollen scale invariant sein
	- Ausführung auf GPU

2. Features extrahieren 
	- verschiedene Ansätze aus der Literatur benennen (SIFT, LBP, IBM)
	
3. Ansätze
	- bag of visual words
	- ML SIFT AE

4. Clustering der Features
	- k-means -> möglichst gpu
	
	
MOVE ANALYSIS
LBP http://bytefish.de/blog/local\_binary\_patterns/
GIST http://cvcl.mit.edu/Papers/IJCV01-Oliva-Torralba.pdf