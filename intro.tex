\chapter{Einleitung}

In dieser Arbeit werden zwei Modelle zur Kategorisierung von großen Bildmengen vorgestellt: Zum Einen das Bag of Visual Words Modell und zum Anderen der Autoencoder. Da die Verarbeitung von solch großen Datenmengen eine hohe Laufzeit aufweist, sollen die wesentlichen Berechnungen parallel durch Grafikkarten ausgeführt werden. Hierfür wurde Nvidias CUDA Plattform als Basis gewählt. Es wurde entschieden zwei verschiedene Verfahren umzusetzen und deren Ergebnisse gegenüber zu stellen. Mit dem Bag of Visual Words wurde ein eher klassisches Verfahren gewählt, dass im Kern mit einem k-means Algorithmus arbeitet. Der Autoencoder ist ein neuronales Netzwerk und damit im Bereich des Machine-Learnings angesiedelt. Damit sind diese Konzepte nicht wirklich neu, sehen gehen bis auf die 1940er Jahre zurück, finden aber nun aufgrund der wachsenden Rechenleistung erst vermehrt Anwendung.

\section{Aufbau und Ablauf}

Das Grundlagenkapitel beginnt mit einer Einführung in digitale Bilder und Features solcher. Hier werden im Weiteren verwendete Notationen und Operationen vorgestellt. Die beiden nächsten Abschnitte widmen sich den Grundlagen des Bag of Visual Words und des Autoencoders. Der Bag of Visual Words nutzt im Wesentlichen einen k-means Algorithmus, deswegen wird hier ein kurzer Überblick über dieses Verfahren gegeben. Autoencoder gehören zur Familie der neuronalen Netze. Aufgrund dessen wird zuerst die Idee die neuronalen Netzen zugrunde liegt betrachtet, bevor die Funktionsweise eines Autoencoders erläutert wird. Um Probleme, die in der Praxis mit Autoencoder auftreten, zu überwinden, werden Erweiterungen des Autoencoders behandelt. Da diese Modelle für die Ausführung auf Grafikkarten entwickelt werden, folgt eine Einführung in Nvidias CUDA Plattform. Nach einer Beschreibung des Ausführungsmodells schließt das Kapitel mit einem Beispiel eines CUDA Programms zur Vektor Addition.

In der Analyse werden zu Beginn etablierte Verfahren zur Feature-Detektion und -Extraktion vorgestellt, um zu entscheiden welcher Detektor bzw. Deskriptor im Weiteren verwendet wird. Da das Verfahren zur Kategorisierung der Bilder unüberwachter Natur sein soll, wird geklärt, wie diese sich zu überwachten Verfahren unterscheiden. Der folgende Abschnitt zum Bag of Visual Words Modell behandelt die Parallelisierbarkeit der Algorithmen die genutzt werden: Auf Basis einer jeweils sequentiellen Variante des k-means und Histogramm Algorithmus wird eine parallele Version hergeleitet, welche auf einer \todo{SPMD} Architektur ausgeführt werden kann.

In der Konzeption werden die Funktionen und der Ablauf beider Programme behandelt. Beim Bag of Visual Words betrifft dies das Aufbauen eines Modells aus den Bild-Features sowie die Klassifizierung eines Bildes anhand des Modells. Für den Entwurf eines Autoencoders wurde das Modell aus der Arbeit von Zhao \todo{[REF]} als Basis verwendet. Da für die Implementierung TensorFlow verwendet wird, wird hier zunächst eine Einführung in das Deep-Learning Framework gegeben und gezeigt, wie neuronale Netze abgebildet werden können. 
\todo{Impl, Results}

\section{Ziele der Arbeit}

Ziel der Arbeit ist es ein Verfahren zu entwickeln, die Bilder einer große Menge in eine vorgegebene Anzahl von Kategorien einzuteilen. Somit kann zum Einen ein Überblick über die vorhandenen Arten von Bildern hergestellt werden und zum Anderen können weitere Bilder auf Basis eines Modells kategorisiert werden. Anforderung an das Modell ist es, dass das Training ohne Lehrer stattfindet, d.h. der Prozess ist unüberwacht. Da die vorliegende Bildmenge bereits über \todo{10.000.000} Bilder enthält, sollen die zeitaufwendigsten Berechnungen parallel durchgeführt werden. Da CUDA in der HsH sowohl gelehrt als auch zu Forschungszwecken genutzt wird, bietet sich Nvidias Plattform an. Darüber hinaus gibt es etablierte Frameworks, wie TensorFlow oder Keras, die CUDA nutzen, dem Entwickler aber eine Abstraktion von dem sehr technischen CUDA C bieten.  

\todo{Vergleich etablierter Verfahren (BoVW / AE)}